\documentclass{article}

\usepackage[T1,T2A]{fontenc}
\usepackage[utf8]{inputenc}
\usepackage[russian]{babel}
\usepackage{amsmath}
\usepackage[a4paper,landscape]{geometry}
\usepackage{algpseudocode}

\author{Виталий Кирсанов}

\begin{document}

\begin{displaymath}
\begin{matrix}
    1 \leq |done| \leq n \\
    \land \\
    1 \leq |last| \leq n \\
    \land \\
    \forall i, 1 \leq i \leq n : |mark_i| = i \\
    \land \\
    \begin{array}{ll}
        last \neq done \implies &\forall i, 1 \leq i \leq n, 8 \leq pc_i < 15 : after_i = done \lor \exists j, 1 \leq j \leq n, 8 \leq pc_j < 15 : |after_i| = j \\
                                &\land \\
                                &\forall i,j, 1 \leq i < j \leq n, 8 \leq pc_i < 15, 8 \leq pc_j < 15 : |after_i| \neq |after_j| \\
                                &\land \\
                                &\exists k, 1 \leq k \leq n, 8 \leq pc_k < 15 : after_k = done \\
    \end{array} \\
    \land \\
    last = done \implies \forall i, 1 \leq i \leq n : pc_i < 7 \lor pc_i \geq 15
\end{matrix}
\end{displaymath}

\begin{algorithmic}[1]
\State $ last \gets done \gets 1 $                  \Comment $ last = done = 1 $
\Procedure{process}{p = 1..n}                       \Comment $ true $
    \State $ mark \gets p $                         \Comment $ mark_p = p $
    \While{$ true $}                                \Comment $ true $
        \State $ mark \gets -mark $                 \Comment $ mark_p \neq last \land mark_p \neq done $
        \State $ after \gets mark $                 \Comment $ after_p = mark_p \land mark_p \neq last \land mark_p \neq done $
        \State \textbf{swap} $ (after, last) $      \Comment $ mark_p \neq after_p \land mark_p \neq done \land last \neq done $
        \If{$ after \neq -mark $}                   \Comment $ |mark_p| \neq |after_p| \land mark_p \neq done \land last \neq done $
            \While{$ after \neq done $}
                \State \textbf{skip}
            \EndWhile                               \Comment $ |mark_p| \neq |after_p| \land after_p = done \land last \neq done $
        \EndIf                                      \Comment $ (|mark_p| \neq |done| \lor -mark_p = done) \land after_p = done \land last \neq done $
        \State \texttt{critical section}
        \State $ done \gets mark $                  \Comment $ |mark_p| \neq |after_p| \lor -mark_p = after_p $
        \State \texttt{noncritical section}
    \EndWhile                                       \Comment $ false $
\EndProcedure
\end{algorithmic}

\begin{enumerate}
    \item Взаимное исключение. Пусть некий поток $ p $ находится в критической секции ($ 13 \leq pc_p < 15 $). Тогда мы
    знаем, что $ last \neq done $ и $ after_p = done $. Пусть теперь некий поток $ q $ пытается войти в критическую
    секцию ($ 7 \leq pc_q < 13 $). Пусть после шага 7 $ after_q = -mark_q $. Но тогда $ after_q $ должно быть равным
    $ done $, что противоречит глобальному инварианту, так как среди всех потоков пытающихся войти или уже вошедших в
    критическую секцию может быть только один, чья переменная $ after $ равняется $ done $. Стало быть после шага 7
    поток $ q $ перейдёт к циклу на шагах 9-11. Выйти из этого цикла поток $ q $ сможет, только лишь когда $ done $
    изменит своё значение и станет равным $ after_q $. Но это возможно лишь после того как поток $ p $, выходя из
    критической секции, изменит значение $ done $. Таким образом никакой поток $ q $ не сможет войти в критическую
    секцию, если она уже занята потоком $ p $.
    \item Отсутствие livelocks. Пусть критическая секция свободна, тогда из глобального инварианта следует, что
    $ last = done $. Пусть далее несколько потоков пытаются сделать шаг 7. Тогда после этого шага найдётся такой поток,
    чья переменная $ after $ будет равняться $ done $, и следовательно, этот поток сможет пройти в критическую секцию,
    так как не задержится в цикле на шагах 9-11.
    \item Отсутствие ненужных задержек доказывается аналогично предыдущему шагу.
    \item При слабосправедливом планировщике голодание потока могло бы возникнуть, если бы этот поток находился вечно в
    цикле на шагах 9-11, то есть его переменная $ after $ никогда бы не приняла значение $ done $.  Пусть некий поток
    $ q $ вошёл во входной протокол раньше нашего голодающего потока $ p $, попал в критическую секцию вышел из неё и
    снова вошёл во входной протокол. Для того чтобы он снова попал в критическую секцию, поток $ after_q $, должен
    установить переменную $ done $ в $ mark_{after_q} $. Короче, я устал и мне лень доводить это до конца. Я и так уже
    больше недели этой задачей занимаюсь.
\end{enumerate}

\end{document}
