\documentclass{article}

\usepackage[T1,T2A]{fontenc}
\usepackage[utf8]{inputenc}
\usepackage[russian]{babel}
\usepackage{amsmath}
\usepackage[a4paper,landscape]{geometry}

\author{Виталий Кирсанов}

\begin{document}
\small

\section{Решение}

Пусть 
\begin{displaymath}
O_q[1:\omega], 1 \leq \omega \leq n 
\end{displaymath}
- все предложения, сделанные женщине \( q \). Каждое предложение представляет собой номер мужчины от \( 1 \) до \( n \). Далее, пусть
\begin{displaymath}
A_p[1:\alpha], 1 \leq \alpha \leq n 
\end{displaymath}
- все ответы, данные мужчине \( p \) женщинами, которым он сделал предложения. Ответ может принимать два значения: \emph{maybe} и \emph{no}.
Пусть
\begin{displaymath}
M_p, 1 \leq p \leq n
\end{displaymath}
- массив предпочтений мужчины \( p \), или его рейтинг женщин. Чем ближе к началу массива находится женщина, тем выше мужчина \( p \) ценит её. Одна
и та же женщина появляется в рейтинге только один раз:
\begin{displaymath}
\forall p,s,t, 1 \leq p \leq n, 1 \leq s < t \leq n : M_p[s] \neq M_p[t]
\end{displaymath}
Из этого следует, что существует однозначная обратная функция \( M_p^{-1} : women \to ratings \). Поэтому свойство упорядоченности массива \( M_p \) по
убыванию рейтинга может быть записано так:
\begin{displaymath}
\forall p,s,t, 1 \leq p \leq n, 1 \leq s < t \leq n : M_p^{-1}[M_p[s]] < M_p^{-1}[M_p[t]]
\end{displaymath}
И наконец, пусть
\begin{displaymath}
W_q, 1 \leq q \leq n
\end{displaymath}
- массив оценок женщиной \( q \) мужчин. Женщина \( q \) не оценивает никаких двух мужчин одинаково:
\begin{displaymath}
\forall q,s,t, 1 \leq q \leq n, 1 \leq s < t \leq n : W_q[s] \neq W_q[t]
\end{displaymath}

Мужчина последовательно делает предложения женщинам в порядке убывания рейтинга до тех пор, пока не получит ответ \emph{maybe} и этот ответ не
будет окончательным, то есть:
\begin{displaymath}
\forall i, 1 \leq i < \alpha : A_p[i] = no \land A_p[\alpha] = maybe
\end{displaymath}

Женщина \( q \) всегда выбирает лучшее предложение, поступившее ей:
\begin{displaymath}
\forall i, 1 \leq i \leq \omega : W_q[e[q]] \geq W_q[O_q[i]] \land \exists ! k : W_q[e[q]] = W[O_q[k]]
\end{displaymath}
Единственность совпадения оценок \( e[q] \) и \( O_q[k] \) следует из того, что ни один мужчина не делает предложение одной и той же женщине
дважды.

Так как всякая женщина в конечном итоге выбирает единственного мужчину, а каждый мужчина никогда не делает предложение двум женщинам одновременно,
всякий мужчина рано или поздно получит окончательный ответ \emph{maybe}. Следовательно процесс нахождения пары в конечном итоге завершится для каждого
мужчины и каждой женщины. Для каждой пары \( (e[i],i), 1 \leq i \leq n \) будет верно следующее:
\begin{equation}
    \label{pr:final1}
    \begin{array}{c}
    \forall j, 1 \leq j < \alpha_{e[i]} : A_{e[i]}[j] = no \\
    \land \\
    A_{e[i]}[\alpha_{e[i]}] = maybe \\
    \land \\
    \forall j, 1 \leq j \leq \omega_i : W_i[e[i]] \geq W_i[O_i[j]] \\
    \land \\
    \exists ! k : W_i[e[i]] = W_i[O_i[k]]
    \end{array}
\end{equation}
Мужчины, получившие отказ, удовлетворяют следующему выражению:
\begin{displaymath}
\begin{array}{c}
\forall p, 1 \leq p \leq n : A_p[i] = no \\
\Leftrightarrow \\
\exists k, 1 \leq k \leq \omega_{M_p[i]}, p \neq O_{M_p[i]}[k] : W_{M_p[i]}[p] < W_{M_p[i]}[O_{M_p[i]}[k]]
\end{array}
\end{displaymath}
А мужчины, получившие ответ \emph{maybe}, удовлетворяют этому выражению:
\begin{displaymath}
\begin{array}{c}
\forall p, 1 \leq p \leq n : A_p[i] = maybe \\
\Leftrightarrow \\
\exists ! k, W_{M_p[i]}[p] = W_{M_p[i]}[O_{M_p[i]}[k]] \land p = O_{M_p[i]}[k]
\end{array}
\end{displaymath}
Тогда \ref{pr:final1} можно записать следующим образом:
\begin{equation}
    \label{pr:final2}
    \begin{array}{c}
    \forall j, 1 \leq j < \alpha_{e[i]}, \exists k, 1 \leq k \leq \omega_{M_{e[i]}[j]}, e[i] \neq O_{M_{e[i]}[j]}[k] : \\
    W_{M_{e[i]}[j]}[e[i]] < W_{M_{e[i]}[j]}[O_{M_{e[i]}[j]}[k]] \\
    \land \\
    \exists ! k : W_i[e[i]] = W_i[O_i[k]] \land e[i] = O_i[k] \land i = M_{e[i]}[\alpha_{e[i]}] \\
    \land \\
    \forall j, 1 \leq j \leq \omega_i : W_i[e[i]] \geq W_i[O_i[j]] \\
    \land \\
    \forall j, \alpha_{e[i]} < j \leq n : M^{-1}_{e[i]}[M_{e[i]}[\alpha_{e[i]}]] < M^{-1}_{e[i]}[M_{e[i]}[j]] \\
    \land \\
    \forall j, \omega_i < j \leq n, \exists k, 1 \leq k \leq \alpha_{O_i[j]}, k \neq M_{O_i[j]}[k] : \\
    M^{-1}_{O_i[j]}[i] > M^{-1}_{O_i[j]}[M_{O_i[j]}[k]]
    \end{array}
\end{equation}

Мы видим, что, несмотря на то, что мужчина \( e[i] \) оценивает женщин \( M_{e[i]}[j], 1 \leq j < \alpha_{e[i]} \) выше, чем
женщину \( M_{e[i}[\alpha_{e[i]}] \), эти женщины предпочли его другим кавалерам, сделавшим им предложение. С другой стороны,
мы видим, что всех женщин \( M_{e[i]}[j], \alpha_{e[i]} < j \leq n \) мужчина \( e[i] \) оценивает ниже, чем женщину  \( i \).
И так как, с одной стороны, мужчина \( e[i] \) получил ответ \emph{maybe} и остановил поиск, а сдругой стороны,
женщина \( i \) ценит мужчину \( e[i] \) выше всех тех, кто сделал ей предложение, а все те, кто не сделал, отдали предпочтение
другим дамам, - то это значит, что пара \( (e[i],i) \) устойчива. А так как логика выбора у всех мужчин и женщин одинакова
соответственно, то искомое утверждение будет истинно.
\begin{displaymath}
\begin{array}{c}
\forall i, j, 1 \leq i < j \leq n : \\
M^{-1}_{e[i]}[i] < M^{-1}_{e[i]}[j] \lor W_j[e[j]] > W_j[e[i]] \\
\land \\
M^{-1}_{e[j]}[j] < M^{-1}_{e[j]}[i] \lor W_i[e[i]] > W_i[e[j]]
\end{array}
\end{displaymath}

Таким образом, если написать программу в соответствии с принципами, описанными в начале доказательства, то эта программа найдёт
\( n \) стабильных пар.

\newpage

Глобальный инвариант:

\begin{displaymath}
\begin{matrix}
(\forall p,i,j, 1 \leq p \leq n, 1 \leq i < j \leq n : M_p[i] \neq M_p[j] \land 1 \leq M_p[i] \leq n) \\
\land {}\\
(\forall q,i,j, 1 \leq q \leq n, 1 \leq i < j \leq n : W_q[i] \neq W_q[j] \land 1 \leq W_q[i] \leq n) \\
\land {}\\
0 \leq c \leq n \\
\land \\
(\forall q, 1 \leq q \leq n : O_q[t_q] = \emptyset \land (\forall i, 0 \leq i < t_q : 1 \leq O_q[i] \leq n) \land (\forall u,v, 0 \leq u < v < t_q : O_q[u] \neq O_q[v])) \\
\land \\
\begin{array}{lll}
(\forall p, 1 \leq p \leq n : {}& 0 \leq r_p \leq n \land {} \\
                            &(r_p > 0 \land a_p \neq \emptyset \implies &(\exists u, 0 \leq u \leq s_{M_p[r_p]} : O_{M_p[r_p]}[u] = p) \land {}\\
                            &                                  &(a_p = no \implies r_p < n \land (\exists v, 0 \leq v \leq s_{M_p[r_p]} : W_{M_p[r_p]}[O_{M_p[r_p]}[v]] > W_{M_p[r_p]}[O_{M_p[r_p]}[u]] )) \land {} \\
                            &                                  &(a_p = maybe \implies (\forall v, 0 \leq v < s_{M_p[r_p]} : W_{M_p[r_p]}[O_{M_p[r_p]}[u]] \geq W_{M_p[r_p]}[O_{M_p[r_p]}[v]])))
\end{array}\\
\land \\
\forall p_1,p_2, 1 \leq p_1 < p_2 \leq n : a_{p_1} = maybe \land a_{p_2} = maybe \implies M_{p_1}[r_{p_1}] \neq M_{p_2}[r_{p_2}] \\
\land \\
\begin{array}{ll}
\forall q_1,q_2, 1 \leq q_1 < q_2 \leq n : {} & e_{q_1} \neq 0 \land e_{q_2} \neq 0 \implies e_{q_1} \neq e_{q_2} \land {}\\
                                              & (\forall i,j, s_{q_1} < i < t_{q_1}, s_{q_2} < j < t_{q_2} : O_{q_1}[i] \neq O_{q_2}[j]) \land {} \\
                                              & O_{q_1}[s_{q_2}] = O_{q_2}[s_{q_2}] \implies (\sigma_{q_1} = true \lor \sigma_{q_2} = true)
\end{array} \\
\end{matrix}
\end{displaymath}

\begin{verbatim}
BEGIN
FOR [ i = 1 .. n ] t[i] = 0
c = 0
FOR [ i = 1 .. n ] LAUNCH man[i]; LAUNCH woman[i]
END
\end{verbatim}
\begin{displaymath}
c = n \land (\forall p, 1 \leq p \leq n : a_p = \emptyset) \land (\forall q, 1 \leq q \leq n : e_q \neq 0 \land s_q = t_q \neq 0 \land M_p[r_p] = q \iff e_q = p) 
\end{displaymath}
\newpage
\begin{verbatim}
PROCESS man [ p = 1 .. n ]
    r = 0
    a = no
\end{verbatim}
\begin{displaymath}
a = no \lor a = maybe
\end{displaymath}
\begin{verbatim}
    DO @@@ psi = psi + 1
        IF a = no
            UNDEF a
            r = r + 1
\end{verbatim}
\begin{displaymath}
\begin{matrix}
\forall \psi_1, \psi_2, \psi_1 \neq \psi_2 : r^{\psi_1} \neq r^{\psi_2} \\
\land \\
\forall q, 1 \leq q \leq n : O_q[s_q] = p \implies \sigma_q = true
\end{matrix}
\end{displaymath}
\begin{verbatim}
          < O[M[r]][t[M[r]]] = p
            t[M[r]] = t[M[r]] + 1 >
\end{verbatim}
\begin{displaymath}
    a = \emptyset \implies (\forall i \neq M[r], 1 \leq i \leq n, O_i[s_i] = p : \sigma_v = true) \land ((\exists ! q, k, q = M[r], s_q < k < t_q : O_q[k] = p) \lor (O_{M[r]}[s_{M[r]}] = p \land \sigma_{M[r]} = false) )
\end{displaymath}
\begin{verbatim}
        ELSE
            UNDEF a
\end{verbatim}
\begin{displaymath}
a = \emptyset \lor a = maybe \lor a = no
\end{displaymath}
\begin{verbatim}
      < AWAIT c = n OR DEFINED a >
\end{verbatim}
\begin{displaymath}
c = n \lor a = maybe \lor a = no
\end{displaymath}
\begin{verbatim}
    WHILE DEFINED a
\end{verbatim}
\begin{displaymath}
c = n
\end{displaymath}
\begin{verbatim}

\end{verbatim}
\newpage
\begin{verbatim}

PROSCESS woman [ q = 1 .. n ]
    e = 0
    W[e] = -INFINITY
    s = 0
\end{verbatim}
\begin{displaymath}
\begin{matrix}
\exists k, 0 \leq k < s_q : O[k] = e \\
\land \\
e \neq 0 \implies a_e = maybe \land c > 0 \\
\land \\
(\forall i, s \leq i < t : a_{O[i]} = \emptyset)
\end{matrix}
\end{displaymath}
\begin{verbatim}
    WHILE TRUE
      < AWAIT c = n OR s < t >
        IF c = n ; BREAK
\end{verbatim}
\begin{displaymath}
s < t
\end{displaymath}
\begin{verbatim}
        IF W[O[s]] > W[e]
            a[e] = no
            a[O[s]] = maybe @@@ sigma = true
            IF e = 0 ; < c = c + 1 >
            e = O[s]
        ELSE
            a[O[s]] = no    @@@ sigma = true
        s = s + 1           @@@ sigma = false
\end{verbatim}
\begin{displaymath}
c = n
\end{displaymath}

\end{document}
