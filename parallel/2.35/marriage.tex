\documentclass{article}

\usepackage[T1,T2A]{fontenc}
\usepackage[utf8]{inputenc}
\usepackage[russian]{babel}
\usepackage{amsmath}

\author{Виталий Кирсанов}

\begin{document}

\section{Решение}

Пусть 
\begin{displaymath}
O_q[1:\omega], 1 \leq \omega \leq n 
\end{displaymath}
- все предложения, сделанные женщине \( q \). Каждое предложение представляет собой номер мужчины от \( 1 \) до \( n \). Далее, пусть
\begin{displaymath}
A_p[1:\alpha], 1 \leq \alpha \leq n 
\end{displaymath}
- все ответы, данные мужчине \( p \) женщинами, которым он сделал предложения. Ответ может принимать два значения: \emph{maybe} и \emph{no}.
Пусть
\begin{displaymath}
M_p, 1 \leq p \leq n
\end{displaymath}
- массив предпочтений мужчины \( p \), или его рейтинг женщин. Чем ближе к началу массива находится женщина, тем выше мужчина \( p \) ценит её. Одна
и та же женщина появляется в рейтинге только один раз:
\begin{displaymath}
\forall p,s,t, 1 \leq p \leq n, 1 \leq s < t \leq n : M_p[s] \neq M_p[t]
\end{displaymath}
Из этого следует, что существует однозначная обратная функция \( M_p^{-1} : women \to ratings \). Поэтому свойство упорядоченности массива \( M_p \) по
убыванию рейтинга может быть записано так:
\begin{displaymath}
\forall p,s,t, 1 \leq p \leq n, 1 \leq s < t \leq n : M_p^{-1}[M_p[s]] < M_p^{-1}[M_p[t]]
\end{displaymath}
И наконец, пусть
\begin{displaymath}
W_q, 1 \leq q \leq n
\end{displaymath}
- массив оценок женщиной \( q \) мужчин. Женщина \( q \) не оценивает никаких двух мужчин одинаково:
\begin{displaymath}
\forall q,s,t, 1 \leq q \leq n, 1 \leq s < t \leq n : W_q[s] \neq W_q[t]
\end{displaymath}

Мужчина последовательно делает предложения женщинам в порядке убывания рейтинга до тех пор, пока не получит ответ \emph{maybe} и этот ответ не
будет окончательным, то есть:
\begin{displaymath}
\forall i, 1 \leq i < \alpha : A_p[i] = no \cap A_p[\alpha] = maybe
\end{displaymath}

Женщина \( q \) всегда выбирает лучшее предложение, поступившее ей:
\begin{displaymath}
\forall i, 1 \leq i \leq \omega : W_q[e[q]] \geq W_q[O_q[i]] \cap \exists ! k : W_q[e[q]] = W[O_q[k]]
\end{displaymath}
Единственность совпадения оценок \( e[q] \) и \( O_q[k] \) следует из того, что ни один мужчина не делает предложение одной и той же женщине
дважды.

Так как всякая женщина в конечном итоге выбирает единственного мужчину, а каждый мужчина никогда не делает предложение двум женщинам одновременно,
всякий мужчина рано или поздно получит окончательный ответ \emph{maybe}. Следовательно процесс нахождения пары в конечном итоге завершится для каждого
мужчины и каждой женщины. Для каждой пары \( (e[i],i), 1 \leq i \leq n \) будет верно следующее:
\begin{equation}
    \label{pr:final1}
    \begin{array}{c}
    \forall j, 1 \leq j < \alpha_{e[i]} : A_{e[i]}[j] = no \\
    \cap \\
    A_{e[i]}[\alpha_{e[i]}] = maybe \\
    \cap \\
    \forall j, 1 \leq j \leq \omega_i : W_i[e[i]] \geq W_i[O_i[j]] \\
    \cap \\
    \exists ! k : W_i[e[i]] = W_i[O_i[k]]
    \end{array}
\end{equation}
Мужчины, получившие отказ, удовлетворяют следующему выражению:
\begin{displaymath}
\begin{array}{c}
\forall p, 1 \leq p \leq n : A_p[i] = no \\
\Leftrightarrow \\
\exists k, 1 \leq k \leq \omega_{M_p[i]}, p \neq O_{M_p[i]}[k] : W_{M_p[i]}[p] < W_{M_p[i]}[O_{M_p[i]}[k]]
\end{array}
\end{displaymath}
А мужчины, получившие ответ \emph{maybe}, удовлетворяют этому выражению:
\begin{displaymath}
\begin{array}{c}
\forall p, 1 \leq p \leq n : A_p[i] = maybe \\
\Leftrightarrow \\
\exists ! k, W_{M_p[i]}[p] = W_{M_p[i]}[O_{M_p[i]}[k]] \cap p = O_{M_p[i]}[k]
\end{array}
\end{displaymath}
Тогда \ref{pr:final1} можно записать следующим образом:
\begin{equation}
    \label{pr:final2}
    \begin{array}{c}
    \forall j, 1 \leq j < \alpha_{e[i]}, \exists k, 1 \leq k \leq \omega_{M_{e[i]}[j]}, e[i] \neq O_{M_{e[i]}[j]}[k] : \\
    W_{M_{e[i]}[j]}[e[i]] < W_{M_{e[i]}[j]}[O_{M_{e[i]}[j]}[k]] \\
    \cap \\
    \exists ! k : W_i[e[i]] = W_i[O_i[k]] \cap e[i] = O_i[k] \cap i = M_{e[i]}[\alpha_{e[i]}] \\
    \cap \\
    \forall j, 1 \leq j \leq \omega_i : W_i[e[i]] \geq W_i[O_i[j]] \\
    \cap \\
    \forall j, \alpha_{e[i]} < j \leq n : M^{-1}_{e[i]}[M_{e[i]}[\alpha_{e[i]}]] < M^{-1}_{e[i]}[M_{e[i]}[j]] \\
    \cap \\
    \forall j, \omega_i < j \leq n, \exists k, 1 \leq k \leq \alpha_{O_i[j]}, k \neq M_{O_i[j]}[k] : \\
    M^{-1}_{O_i[j]}[i] > M^{-1}_{O_i[j]}[M_{O_i[j]}[k]]
    \end{array}
\end{equation}

Мы видим, что, несмотря на то, что мужчина \( e[i] \) оценивает женщин \( M_{e[i]}[j], 1 \leq j < \alpha_{e[i]} \) выше, чем
женщину \( M_{e[i}[\alpha_{e[i]}] \), эти женщины предпочли его другим кавалерам, сделавшим им предложение. С другой стороны,
мы видим, что всех женщин \( M_{e[i]}[j], \alpha_{e[i]} < j \leq n \) мужчина \( e[i] \) оценивает ниже, чем женщину  \( i \).
И так как, с одной стороны, мужчина \( e[i] \) получил ответ \emph{maybe} и остановил поиск, а сдругой стороны,
женщина \( i \) ценит мужчину \( e[i] \) выше всех тех, кто сделал ей предложение, а все те, кто не сделал, отдали предпочтение
другим дамам, - то это значит, что пара \( (e[i],i) \) устойчива. А так как логика выбора у всех мужчин и женщин одинакова
соответственно, то искомое утверждение будет истинно.
\begin{displaymath}
\begin{array}{c}
\forall i, j, 1 \leq i < j \leq n : \\
M^{-1}_{e[i]}[i] < M^{-1}_{e[i]}[j] \cup W_j[e[j]] > W_j[e[i]] \\
\cap \\
M^{-1}_{e[j]}[j] < M^{-1}_{e[j]}[i] \cup W_i[e[i]] > W_i[e[j]]
\end{array}
\end{displaymath}

Таким образом, если написать программу в соответствии с принципами, описанными в начале доказательства, то эта программа найдёт
\( n \) стабильных пар.

\begin{verbatim}
BEGIN
@@@ FOR [ i = 1 .. n ] ai[i] = 0; oi[i] = 1

c = 0
FOR [ i = 1 .. n ] UNDEF O[i][oi[i]]
\end{verbatim}
\begin{displaymath}
\begin{array}{c}
c = 0 \cap \forall i, 1 \leq i \leq n : ai[i] = 0 \cap oi[i] = 1 \cap O[i][oi[i]] = \emptyset
\end{array}
\end{displaymath}
\begin{verbatim}
FOR [ i = 1 .. n ] LAUNCH man[i]; LAUNCH woman[i]
END

PROCESS man [ p = 1 .. n ]
    r = 0
    a[p] = no @@@ A[p][0] = no
\end{verbatim}
\begin{displaymath}
\begin{array}{c}
r = ai[p] \\
\cap \\
\forall i, 0 \leq i < r : A[p][i] = no \cap (A[p][r] = no \cup A[p][r] = maybe)\\
\cap \\
a[p] = A[p][r] \cap A[p][r + 1]] = \emptyset \\
\cap \\
0 \leq c \leq n \\
\cap \\
\forall i, 1 \leq i \leq n : O[i][oi[i]] = \emptyset\\
\end{array}
\end{displaymath}
\begin{verbatim}
    DO
        @@@ sai = r
        # sai = save answer index
        IF a[p] = no
            UNDEF a[p]  @@@ UNDEF A[p][ai[p]]
            r = r + 1
\end{verbatim}
\begin{displaymath}
r = ai[p] + 1
\end{displaymath}
\begin{verbatim}
          < O[m[p][r]][oi[m[p][r]]] = p
            @@@ soi = oi[m[p][r]]
            # soi = save offer index
            oi[m[p][r]] = oi[m[p][r]] + 1
            UNDEF O[m[p][r]][oi[m[p][r]]] >
\end{verbatim}
\begin{displaymath}
\begin{array}{c}
A[p][sai] = no \cap sai + 1 = r \\
\cap \\
O[m[p][r]][soi] = p \\
\cap \\
(a[p] = \emptyset \cap r = ai[p] + 1 \cup (a[p] = maybe \cup a[p] = no) \cap r = ai[p]) \\
\cap \\
a[p] = A[p][r] \\
\cup \\
A[p][sai] = maybe \cap sai = r
\end{array}
\end{displaymath}
\begin{verbatim}
      < AWAIT c = n OR DEFINED a[p] >
\end{verbatim}
\begin{displaymath}
\begin{array}{c}
r = ai[p] \cap a[p] = A[p][r] \\
\cap \\
c < n \cap (a[p] = maybe \cup a[p] = no) \cup c = n \cap a[p] =  maybe
\end{array}
\end{displaymath}
\begin{verbatim}
    WHILE c < n
\end{verbatim}
\begin{displaymath}
\begin{array}{c}
r = ai[p] \\
\cap \\
\forall i, 0 \leq i < r : A[p][i] = no \cap A[p][r] = maybe\\
\cap \\
a[p] = A[p][r] \cap A[p][ai[p] + 1] = \emptyset \\
\cap \\
c = n \\
\cap \\
\forall i, 1 \leq i \leq n : O[i][oi[i]] = \emptyset\\
\end{array}
\end{displaymath}
\begin{verbatim}

PROSCESS woman [ q = 1 .. n ]
    @@@ W[q][0] = -INFINITY
    e[q] = 0
    r = 1
\end{verbatim}
\begin{displaymath}
\begin{array}{c}
\forall i, 1 \leq i < r : W[q][O[q][i]] \leq W[q][e[q]] \\
\cap \\
\exists ! k, 1 \leq k < r : W[q][O[q][k]] = W[q][e[q]] \\
\cap \\
O[q][oi[q]] = \emptyset \\
\cap \\
r \leq oi[q] \\
\cap \\
0 \leq c \leq n
\end{array}
\end{displaymath}
\begin{verbatim}
    WHILE TRUE
      < AWAIT c = n OR DEFINED O[q][r] >
\end{verbatim}
\begin{displaymath}
\begin{array}{c}
c = n \cap r = oi[q] \\
\cup \\
(c < n \cap O[q][r] \neq \emptyset \\
\cap \\
\forall i, 1 \leq i \leq r : W[q][O[q][i]] \leq W[q][e[q]] \cup W[q][O[q][i]] > W[q][e[q]])
\end{array}
\end{displaymath}
\begin{verbatim}
        IF c = n
            RETURN
        IF W[q][O[q][r]] > W[q][e[q]]
\end{verbatim}
\begin{displaymath}
\begin{array}{c}
a[O[q][r]] = \emptyset
\end{array}
\end{displaymath}
\begin{verbatim}
            IF e[q] = 0
              < c = c + 1 >
            a[O[q][r]] = maybe
            @@@ ai[O[q][r]] = ai[O[q][r]] + 1
            @@@ A[O[q][r]][ai[O[q][r]]] = maybe
            @@@ UNDEF A[O[q][r]][ai[O[q][r]] + 1]
\end{verbatim}
\begin{displaymath}
\begin{array}{c}
a[e[q]] = maybe
\end{array}
\end{displaymath}
\begin{verbatim}
            a[e[q]] = no
            @@@ A[e[q]][ai[e[q]]] = no
            e[q] = O[q][r]
        ELSE
\end{verbatim}
\begin{displaymath}
\begin{array}{c}
a[O[q][r]] = \emptyset
\end{array}
\end{displaymath}
\begin{verbatim}
            a[O[q][r]] = no
            @@@ ai[O[q][r]] = ai[O[q][r]] + 1
            @@@ A[O[q][r]][ai[O[q][r]]] = no
            @@@ UNDEF A[O[q][r]][ai[O[q][r]] + 1]
        r = r + 1
\end{verbatim}
\begin{displaymath}
\begin{array}{c}
\forall i, 1 \leq i < r : W[q][O[q][i]] \leq W[q][e[q]] \\
\cap \\
\exists ! k, 1 \leq k < r : W[q][O[q][k]] = W[q][e[q]] \\
\end{array}
\end{displaymath}
\begin{verbatim}


\end{verbatim}
\begin{displaymath}
\begin{array}{c}
\forall i, 1 \leq i < r : W[q][O[q][i]] \leq W[q][e[q]] \\
\cap \\
\exists ! k, 1 \leq k < r : W[q][O[q][k]] = W[q][e[q]] \\
\cap \\
O[q][r] = \emptyset \\
\cap \\
c = n \\
\cap \\
r = oi[q]
\end{array}
\end{displaymath}

\end{document}
